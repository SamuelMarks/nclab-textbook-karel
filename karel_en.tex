\documentclass[article,A4,12pt]{llncs}

% Conditional compilation.
% NOTE: If you set fullversionfalse, just compile ONCE so that TOC stays unchanged.
\newif\iffullversion
\fullversiontrue
%\fullversionfalse

\usepackage[T1]{fontenc}
\usepackage{amsmath}
\usepackage{amssymb}
\usepackage{color}
\usepackage{amsfonts}
\usepackage{mathrsfs, bm}

\usepackage{graphicx}
\usepackage{tabularx}
\usepackage{subfig}
\usepackage{epsf,times}
\usepackage{color}
\usepackage{wrapfig}
\usepackage{cases}
\usepackage{multicol}

\usepackage{palatino}

\usepackage[T1]{fontenc}
%\newcommand{\tmname}[1]{\textsc{#1}}
%\newcommand{\tmop}[1]{\ensuremath{\operatorname{#1}}}
%\newcommand{\tmsamp}[1]{\textsf{#1}}
%\newcommand{\tmtextsc}[1]{{\scshape{#1}}}
%\newcommand{\tmtextsl}[1]{{\slshape{#1}}}
%\newcommand{\tmtexttt}[1]{{\ttfamily{#1}}}

\leftmargin=0.0cm
\oddsidemargin=0.5cm
\evensidemargin=0.5cm
\topmargin=0cm
\textwidth=16.0cm
%\textheight=21.5cm
\textheight=20.0cm
\pagestyle{plain}
\setlength{\columnsep}{20pt}

\def\m{\mathbf{m}}
\def\H{\mathbf{H}}
\def\E{\mathbf{E}}
\newcommand{\vepsi}{{\varepsilon}}
\def\hnorm#1#2{\vert\,#1\,\vert_{#2}}
\newcommand{\R}{{\mathbb R}}
\newcommand{\Sph}{{\mathbb S}}
\def\x{\mathbf{x}}
\def\hvec{\overline{\mathbf{h}}}
\def\evec{\overline{\mathbf{e}}}

\newcommand{ \etal}{\mbox{\emph{et al. }}}

\newcommand\vect[1]{\mbf{#1}}
\newcommand{\mbf}[1]{\mbox{\boldmath$#1$}} 
\newcommand{\RC}[1]{#1 $\times$ #1 $\times$ #1}
\def\um{$\mu$m}
\def\C{$^{\circ}\mathrm{C}$}

\newcommand{\Rmnum}[1]{\expandafter\@slowromancap\romannumeral #1@}

% DEFINITION OF CUSTOM FONT SIZE
\newcommand{\customfontA}{\fontsize{50}{55}\selectfont}
\newcommand{\customfontB}{\fontsize{14.4}{20}\selectfont}
\newcommand{\customfontC}{\fontsize{30}{35}\selectfont}

\DeclareMathAlphabet{\mathpzc}{OT1}{pzc}{m}{it}

\def\clovek#1{\noindent\bgroup\vbox{\noindent#1}\egroup\vskip1em}

% TO INPUT BACKGROUND IMAGE
%\usepackage{eso-pic}
%\newcommand\BackgroundPic{
%\put(0,0){
%\parbox[b][\paperheight]{\paperwidth}{
%\vfill
%\centering
%\includegraphics[width=\paperwidth,height=\paperheight]{img/karel-frontpage.png}
%%\includegraphics[width=\paperwidth,height=\paperheight]{img/background.jpg}
%\vfill
%}}}

\usepackage{fancyvrb}

\newenvironment{bluecode}{\VerbatimEnvironment \color{blue} \begin{Verbatim}}
{\end{Verbatim}}


\begin{document}

% INPUTTING BACKGROUND IMAGE
%\AddToShipoutPicture{\BackgroundPic}
%\vbox{}
%\pagestyle{empty}
%\newpage
%\textwidth=15.5cm
%\ClearShipoutPicture
%\newpage

%%%%%%%%%%%%%%%%%%%%%%%%%%%%%%%%%%%%%%%%%%%%%%%%%%%%%%%%%%%%%%%%%%%%%%%%%
\pagestyle{empty}

\vbox{}
\begin{figure}[!ht]
%\hspace{-4mm}
\includegraphics[width=8cm]{imgp/logo.png}
\vspace{10mm}
\end{figure}
\vbox{}
\vspace{1.5cm}

\begin{center}
{\huge \bf First Course in Programming}
\end{center}

\begin{figure}[!ht]
\begin{center}
\vspace{-6mm}
\includegraphics[width=0.26\textheight]{imgk/karel-logo.png}\ \ \ \ \ \ 
\includegraphics[width=0.2\textheight]{imgp/python-logo.png}
\vbox{}
\vspace{-9mm}
\end{center}
\end{figure}
\begin{center}
{\huge \bf with Karel the Robot and Python}
\end{center}
\vbox{}
\vspace{5mm}
\begin{center}
\iffullversion
\else
\centerline{\huge \color{red}{PREVIEW}}
\fi
\vfill
%{\large
%{\bf Pavel Solin \& Salih Dede}
%Contribute and become a co-author!
%}
\end{center}
\vfill
\vfill
\begin{center}
Revision Sept-22-2012. Copyright 2012 FEMhub Inc. All rights reserved.
\end{center}
\newpage
\vbox{}
\vfill
{
\noindent
{\bf About this Textbook}\\[4mm]
This free textbook is provided as a courtesy to NCLab users. 
It will help you discover the elegance and power of procedural and 
object-oriented programming with Karel the Robot and Python. The former 
is a famous educational language created and still used at the Stanford University, 
the latter is a modern high-level dynamical programming language that is used
in many areas of business, engineering, and science today. After taking 
this course, you will have solid theoretical knowledge and vast
practical experience with computer programming. \\[4mm]

\noindent
{\bf About the Authors}\\[4mm]
Pavel Solin is Professor of Computational Science at the University of Nevada, Reno. 
Computer programming has been his deep personal passion since childhood, and later it 
became part of his profession. He has been developing advanced computer programs in many
different languages, supervising large software projects, and teaching programming to 
college-level students. Salih Dede is Instructor of Computer
Programming at the Coral Academy of Science High School in Reno. He has vast experience 
with teaching programming to high school students, and outstanding instructional
achievements. \\[4mm]

\noindent
{\bf Become a Co-Author}\\[4mm]
This book is meant to become a community project. All contributors become automatically
co-authors. The book needs more illustrations, as well as more advanced exercises 
for Karel and Python. If you are interested in contributing, let know Pavel at 
{\tt pavel@femhub.com}.\\[4mm]


\noindent
{\bf For Instructors}\\[4mm]
Review Book and Exercise Book containing 
hundreds of review questions with answers and programming exercises with
solutions, respectively, are part of the NCLab-powered course 
{\em Intro to Programming with Karel the Robot and Python} that is 
available at \\

{\color{blue}
\centerline{\tt http://introtoprogramming.net}
}
\vspace{5mm}

\noindent
for a small subscription fee.  In addition to 
standard cloud benefits such as anytime-anywhere accessibility through 
the web browser interface and working from mobile devices, the course comes with 
additional attractive features including 
automated grading and student progress tracking. 
The course will be opened in January 2013.
}
\vfill



\newpage
%{\ }
\setcounter{tocdepth}{2}
\tableofcontents
%\pagestyle{plain}

\newpage

\pagestyle{plain}
\setcounter{page}{1}

%%%%%%%%%%%%%%%%%%%%%%%%%%%%%%%%%%%%%%%%%%%%%%%%%%%%%%%%%%%%%%%%%%%%%%%%%
\pagestyle{plain}
\setcounter{page}{1}
\section*{Foreword}
This course provides a gentle yet efficient and comprehensive introduction to modern algorithmic 
design and computer programming. It consists of two programming languages -- {\em Karel the Robot} 
and {\em Python}. Karel the Robot is a famous educational programming language that was created 
at the Stanford University. It features a robot that collects gems in a maze, and helps students 
develop advanced algorithmic skills using a handful of simple commands that do not involve math. 
Python is a modern high-level dynamical programming language that is widely used in business, 
science, and engineering applications. It is much easier to learn than C, C++ or Java.

Starting with Karel the Robot, students discover principles of computer programming effortlessly,
without being exposed to technical details of conventional programming languages.
Could such an introduction be done with a conventional programming language? Yes, but 
it would be obscurred and slowed down by technicalities, not mentioning that it would 
not be fun. The course starts out in Manual mode (Level 0) where the robot can be guided via 
clicking on five buttons {\em Go} (make one step forward), {\em Left} (turn left), {\em Right} 
(turn right), {\em Put} (put a gem on the groud) and {\em Get} (pick up a gem from the ground). 

\begin{figure}[!ht]
\begin{center}
\includegraphics[width=0.75\textwidth]{imgk/fore-1.png}
\end{center}
\vspace{-2mm}
\caption{Sample game {\em Diamond Mine} in Manual mode.}
\label{fig:f1}
\vspace{-4mm}
\end{figure}
\noindent
In the next level which is called {\em Bridge to Programming} students keep solving 
problems by typing the commands {\tt go}, {\tt left}, {\tt right}, {\tt put} and {\tt get} 
instead of clicking on buttons. 

The need for higher functionality such as loops, conditions, and custom commands arises 
naturally as game goals become more complicated. 
Students learn quickly that it is advantageous to break complex tasks into smaller 
ones, which is one of the most important principle of computer 
programming. The textbook is written by programming experts, and in addition to 
advanced programming skills the students gain an overview of good and bad 
programming habits.

\begin{figure}[!ht]
\begin{center}
\includegraphics[width=0.75\textwidth]{imgk/fore-2.png}
\end{center}
\vspace{-2mm}
\caption{Sample game {\em Speleologist} in Programming mode.}
\label{fig:f2}
\vspace{-4mm}
\end{figure}
\noindent
The syntax of Karel the Robot is very close to Python -- in fact Python feels like Karel's
older brother. The transition from Karel to Python is as seamless as the transition from 
one Karel's Level to another. 
In Python, students continue building their skills in the same way they did with Karel. 
They learn more advanced concepts including mathematical operations, plotting, local and 
global variables, strings, tuples, lists, dictionaries, and others. They get to solve 
realistic problems, but Python is still a lot of fun. 

A strong companion of Python are its libraries. Besides the Standard Library that
contains many build-in functions not present in lower-level languages such as 
Java, C, C++ or Fortran, Python also has powerful scientific libraries including 
Scipy, Numpy, Matplotlib, Pylab, Sympy and others. With these, students are ready to 
solve entry-level scientific and engineering problems. These libraries are discussed
at the end of the course.  

\part{Karel the Robot}

\input part-1.tex

\part{Introduction to Python}

\input part-2.tex

\section{What next?}

Congratulations, you made it! We hope that you enjoyed the textbook and the 
exercises. If you can think of any way to improve the application Karel the 
Robot or this tutorial, we would be very happy to hear from you. If you 
have an interesting new game or exercises for Karel, please let us know as well. 

Although you may feel like an Almighty Programmer right now, we would
recommend staying humble. Even the most experienced programmers are
learning new things all the time. There is much more to Python that 
we managed to cover in this introductory textbook. You already know 
about the Internet resources where you can learn more.  

Alternatively, you may dive into a next programming language! We would 
recommend Javascript since this is the most popular language for web 
development. Of course there are many more languages to explore, including 
C/C++, Java, Perl, Ruby, Lua and others.\\

\noindent
In any case, our team wishes you good luck, and keep us in your 
favorite bookmarks! \\

\hbox{} \hfill{} Your NCLab Team




\end{document}
